\documentclass[12pt,a4paper]{article}

\usepackage[utf8]{inputenc}
\usepackage[english]{babel}
\usepackage{amssymb, amsmath}
\usepackage{fullpage}
\usepackage{parskip}
\usepackage{url}
\usepackage[dvipsnames]{xcolor}
\usepackage{enumitem}
\usepackage{graphicx}

\renewcommand{\vec}[1]{\mathbf{#1}}
\newcommand{\from}{\colon}
\newcommand{\IR}{\mathbb{R}}
\newcommand{\IN}{\mathbb{N}}
\newcommand{\norm}[1]{\left\|#1\right\|}

\begin{document}
    
    %%%%%%%%%%%%%%%%%%%%%%%%%%%%%%%%%%%%%%%%%%%%%%%%%%
    On a question with containers
    \hfill
    RA, \today
    %%%%%%%%%%%%%%%%%%%%%%%%%%%%%%%%%%%%%%%%%%%%%%%%%%
    
	\subsection*{Problem setup}
    
	There are three water containers.
	%
	Containers \#1 and \#2 have outflows into \#3.
	%
	Container \#2 is constantly fed through a tap.
	%
	Container \#3 is being drained through an opening.
	%
	What are the water levels in the containers?
	
	\subsection*{Notation}
	
	At any given time, for the container $k = 1, 2, 3$,
	\begin{itemize}
	\item 
		the water level is $h_k$ [$\text{m}$],
	\item
		the water volume is $V_k$ [$\text{m}^3$],
	\item
		the container capacity is $V_k^\text{max}$ [$\text{m}^3$],
	\item
		the outflow is $y_k \geq 0$ [$\mathrm{m^3/s}$]
		and
		the inflow is $x_k \geq 0$ [$\mathrm{m^3/s}$].
	\end{itemize}
	%
	
	\subsection*{ODE}
	
	This is sufficient to formulate the following ODEs.
	%
	As long as no container is maximally filled, we have
	\begin{subequations}
		\begin{align}
			\tfrac{d}{dt} V_1 & = -y_1 , \\
			\tfrac{d}{dt} V_2 & = -y_2 + x_2 , \\
			\tfrac{d}{dt} V_3 & = -y_3 + y_1 + y_2 .
		\end{align}
	\end{subequations}
	
	More generally,
	if a container is maximally filled
	then
	the water volume is allowed to decrease but not increase,
	thus
	(the first $\min$ is redundant)
	\begin{subequations}
		\begin{align}
			\tfrac{d}{dt} V_1 & = \min\{ 0, -y_1 \} , \\
			\tfrac{d}{dt} V_2 & = \min\{ 0, -y_2 + x_2 \} , \\
			\tfrac{d}{dt} V_3 & = \min\{ 0, -y_3 + y_1 + y_2 \} .
		\end{align}
	\end{subequations}
	%
	Here
	we anticipate the physically meaningful setup
	that
	an empty container yields no outflow,
	while
	a full container simply spills over water that is lost
	rather than flowing into one of the other containers.
	
	\subsection*{Assumptions}
	
	The following are reasonable assumptions
	but subject to discussion.
	%
	\begin{itemize}
	\item
		All free water surfaces are in contact with the atmosphere
		and the temperature is room temperature throughout.
	\item
		Container $k$ has a constant horizontal cross-section 
		of area $A_k$ [$\text{m}^2$].
	\item
		The tap supply of container \#2
		is at a constant rate $x_2 \geq 0$ [$\mathrm{m^3 / s}$].
	\item
		We neglect the initiation time
		when any valves are opened,
		i.e.~we start with a fully developed flow.
	\item
		The flows $y_1$ and $y_2$ are unregulated flows
		due to geometry and gravity only.
		%
		The opening is merely an orifice or at most 
		a short pipe pointing straight down out of 
		the bottom of the container
		and
		of cross-section $a_k$ [$\mathrm{m^2}$]
		that is relatively thin
		compared to the container dimensions.
	\item
		Water density is constant at $\rho$ [$\mathrm{kg/m^3}$].
		%
		The graviational acceleration is $g$ [$\mathrm{m/s^2}$].
		
	\item
		Friction losses at walls and due to viscosity are ignored.
		%
		Standard engineering assumptions of incompressible flow,
		in particular
		there is 
		(negligible flow and)
		static pressure of $(g \rho h_k + \mathrm{1atm})$ in the container 
		at the orifice
		and
		$\mathrm{1atm}$
		just outside the orifice;
		by the Bernoulli equation of flow, this translates
		into an outflow velocity $v_k$
		given by
		$\frac12 \rho v_k^2 = g \rho h_k$.
		%
		In reality, the flow is smaller due to ``geometrical conditions''
		(cf.~discharge coefficient).
		
		The same outflow velocity is obtained if we simply
		assume that
		the water column above the orifice 
		falls straight through the opening without any friction
		while being ``refilled'' 
		from the top water layer of the container.
	\end{itemize}
	
	\subsection*{Closure of the ODE system}
	
	The ODE system can now be supplemented 
	and
	fully specified
	(up to initial conditions)
	with the following
	equations
	($k = 1, 2, 3$):
	\begin{align}
		\text{water volume}: \quad
		V_k & = A_k h_k,
		\\
		\text{outflow}: \quad
		y_k & = a_k v_k = a_k \sqrt{ 2 g h_k }
		.
	\end{align}

% 	\subsection*{Solution}
% 	
% 	The full ODE system probably does not admit 
% 	a closed-form analytic solution,
% 	but
% 	the first equation $\tfrac{d}{dt} V_1 = -y_1$
% 	does.
% 	%
% 	After simplification, we have
% 	$\tfrac{d}{dt} h_1 = -c_1 \sqrt{h_1}$
% 	where $c_1 = \frac{a_k}{A_k} \sqrt{ 2 g }$.
\end{document}
